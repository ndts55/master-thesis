\chapter{Introduction}
% Brief explanation of what CUDAjectory does
% 	What does CUDAjectory do?
% 		software that is able to propagate the trajectories of many samples in parallel on a CUDA-capable GPU \cite{geda_massive_2019}
% 	initially authored by GEDA as part of a masters thesis at the TU Delft
The design and analysis of satellite orbits forms an integral part of every space project.
This analysis requires fast, high fidelity, and accurate trajectory simulations for multiple samples.
For some astrodynamic applications propagation times can be several thousands or millions of years and often involve large sample sizes.
Running simulations this large on CPUs can take a significant amount of time.
To this end, \citeauthor{geda_massive_2019} developed \cudaj in \citeyear{geda_massive_2019}\cite{geda_massive_2019}.
\cudaj enables fast trajectory simulations of many samples in parallel on a CUDA-capable GPU.
\cudaj was recently rewritten by researchers at the \esoc.
We will refer to the previous version as \cudaj{1} and to the new rewritten version as \cudaj{2}.

% What are we doing?
% 	Analyzing the performance characteristics of the CUDAjectory v2
% 		Pinpoint performance bottlenecks / suboptimal program behavior
% 			By using profiling tools like NSight Compute / Systems
% 		Determine why these performance bottlenecks occur
% 	Improve performance of CUDAjectory v2 by fixing these performance bottlenecks
% 	To this end we start out by introducing CUDA / MPP / performance analysis concepts before analyzing CUDAjectory v2
% 	Then we walk through the changes made to the code and how they impacted performance
% 	Finally, we present how the overall performance compares to the initial codebase
For this work we intend to analyze the performance of \cudaj{2}, pinpoint performance bottlenecks and areas that can be improved.
Then, we improve the program performance, utilizing the information gathered in the analysis.

In the following sections, we will introduce all the necessary concepts of \cuda programming, performance metrics, the tools with which they can be gathered and what they can tell us about a programs performance behavior.
Then, we explore the structure of \cudaj{2} and how it functions.

% TODO write more about the performance analysis once I have written an actual analysis
Using this information we then analyze the performance of \cudaj{2}.
We pinpoint performance bottlenecks and areas where it can be improved.

% TODO write more about improving the performance once I have actually improved the performance
Following the results of the performance analysis, we make changes to the source code and show how those changes impact performance.

Our goal is to improve overall performance by increasing efficient resource usage to enable larger-scale and faster trajectory simulations to aid in the design of space missions.

% What are we hoping to achieve?
% 	Improved performance of CUDAjectory v2 to enable larger-scale and faster trajectory computations
% Why do we want to achieve it?
% 	CUDAjectory already represents a huge improvement in terms of speedup compared to traditional methods which use the CPU (Godot)
% 	CUDAjectory v2 also already surpasses the performance of its predecessor
% 	Our aim is to improve CUDAjectory v2 even further
% 	TODO does this actually answer WHY?

\section{\cuda}
% TODO include some figures, use them to explain
The Compute Unified Device Architecture (\cuda) is a general-purpose parallel computing platform, providing hardware and sofware components that allow developers to write and execute highly parallelized applications.
Unlike traditional \cpu{s}, \cuda devices posses a larger number of processing units that can execute code in parallel.
Admittedly, the processing units in a \cuda device are typically less powerful than modern \cpu{} cores but certain workloads benefit hugely from the ability to perform a large number of calculations in parallel.

In this section we briefly introduce relevant concepts of \cuda that enable massively parallel program execution.

\subsection{Programming Model}
In \cuda, developers organize parallel code execution with grids and blocks.
These are used to organize threads, which execute kernels on \cuda devices.

A grid is a collection of blocks.
It provides the overall structure for organizing blocks.

Each block has a unique identifier within a grid, given by \texttt{blockIdx}, which is a 3D-index\cite{hwu_programming_2023}.
Threads within a block can synchronize their execution and access the same shared memory.

Each thread has a unique identifier within a block, given by \texttt{threadIdx}, which is a 3D-index\cite{hwu_programming_2023}.
One thread executes a single instance of a kernel function.

Kernels are special functions that are executed on \cuda devices.
Within a kernel, developers have access to the thread index, the block index, the block dimensions, and the grid dimensions.
Kernels can be launched from the host on the device.
When launching a kernel, the grid dimensions and the block dimensions have to be specified\cite{hwu_programming_2023}.

% TODO introduce execution ordering with streams, graphs, events

% Streams
% 	  a queue of GPU operations that are executed in order
% 	  multiple streams allow developers to schedule GPU operations concurrently
% 	  streams are independent of each other, there is no implicit ordering between GPU operations on different streams

% kernel in the same stream are executed in the order they are launched
% there is no implicit ordering between GPU operations on different streams
% explicit ordering is possible through the use of events (maybe out of scope)

\subsection{Compute \& Memory Resources}
% TODO recreate figure from hwu_programming_2023 figure 4.1
\cuda devices are organized into an array of Streaming Multiprocessors (\sm)\cite{hwu_programming_2023}.
A modern \cuda device, such as the \nvidia H100, has up to 144 \sms\cite{elster_nvidia_2022}.
Each \sm has multiple processing units called \cuda cores that share control logic and on-chip memory resources.
% TODO find source for how many cuda cores each sm in an h100 has
% TODO find out more information about what is meant by "Control"

The memory resources are comprised of a register file, local memory, and shared memory.

The register file contains the registers that can only be accessed by the owning thread.
To access the data in a register no memory operations are necessary, thus providing fast low-latency memory to each executing thread.
Should a thread require too many registers the data spills into local memory, which does require memory operations to access, thus decreasing performance.

Shared memory can be accessed by all threads in a block and require memory operations to read from and write to it.
It is meant to store shared data for multiple threads and designed to handle multiple memory accesses simultaneously.
This is achieved by spreading the memory addresses across memory-banks, each of which can serve one access per cycle.

All \sms are connected to global memory, which resides in on-device DRAM.
It tends to have long access latency but provides the largest amount of memory.

\subsection{Execution of a CUDA program}
% TODO memory op coalescing
% TODO memory access through L1 to L2?

To launch a kernel the developer has to specify the grid and block dimensions.
Additionally, the size of shared memory per block and the stream identifier can be given\cite{hwu_programming_2023}.

During kernel execution each block is assigned to one \sm.
This enables threads within the same block to synchronize their execution and access the same shared memory.
The shared memory of each block resides entirely in the on-chip memory the \sm provides\cite{hwu_programming_2023}.

For execution, the \sm schedules warps of threads from a block, which are groups of 32 threads and represent the fundamental unit of scheduling in \cuda.
Each thread is executed on one \cuda core.
Threads in the same warp are issued the same instructions, to be executed in lockstep.
This is most efficient, when all threads execute the same instruction, \ie they follow the same control flow path.
Otherwise, the warp has divergence, meaning that some threads in a wapr are idle because they followed a different control flow path\cite{hwu_programming_2023}.

Ideally, most, if not all, cores in an \sm are occupied executing threads.
This is only possible when the resources an \sm can provide, such as registers, shared and local memory, and cores, is sufficient for all scheduled threads.
If that is not the case, \eg the number of registers a thread uses or the amount of shared memory a thread requires is too high, the \sm can not occupy all cores which remain unused\cite{hwu_programming_2023}.


% Introduce performance metrics
% 	TODO What are they?
% 	TODO What do they tell us about the program?
% 	TODO How do we collect them?

% TODO Maintain a list of all the performance improvement techniques that we use throughout the thesis

% introduce CUDAjectory
% 	performs trajectory calculations
% 		take many samples
% 		Because the samples are independent of each other we can propagate each one in parallel
% 		Godot does this on the CPU
% 		CUDAjectory does this on the GPU
% TODO indirect speec for claims about how bad CUDAjectory v1 is
% 	CUDAjectory v2 is a rewrite of CUDAjectory v1
% 		authored by scientist at European Space Operations Centre (ESOC)
% 		v1 was a successful prototype
% 		issues with CUDAjectory v1
% 			monolithic software
% 				limited maintainability
% 				difficult to extend
% 			little testing
% 			low robustness level on memory management
% 			cumbersome API
% 		rewrite with focus on maintainability, extensibility, UX, DX, and performance
% 	How is it structured? (v2)
% 		modular architecture
% 			TODO describe modular architecture of CUDAjectory v2
% 			Mapping function, Mapper, Mathematical model
% 		Fast Expression Template Algebra (FETA)
% 			implements vectors, arrays of vectors
% 			simplifies GPU programs by encapsulating most of its complexity
% 			TODO Is it similar to CuPy and cuBLAS? What are they?
% 		Parallel Algorithms for Real-time Mathematics (PARM)
% 			relies on FETA array and vector array types
% 			provides more advanced algorithms for scientific computing
% 				interpolation
% 					used by BRIE for ephemeris computation
% 					chebyshev polynomials
% 				root finding
% 				integration
% 					used by CUDAjectory
% 					numerical simulation schemes (e.g., Runge Kutta)
% 				statistics, actually not developed yet
% 		Body-Referenced Interplanetary Ephemeris (BRIE)
% 			relies on FETA array and vector array types
% 			simplifies ephemeris management
% 			rearranges SPICE data to be more efficient on GPUs
% 			works on CPUs and GPUs
% 			TODO explain what ephemeris and SPICE are? Maybe we can circumvent these explanations?
% 		CUDAjectory
% 			Provides
% 				the actual mathematical model
% 				physical models
% 				environmental models
% 				events
% 				interactions
% 				simulation samples / particles
% 			Python module for easy usage
% 			TODO What else does CUDAjectory provide to the calculation?

% TODO Compare CUDAjectory performance with Godot performance to explain why CUDAjectory uses CUDA

% Introduce what we're doing
% 	TODO What are we doing?
% 		performance analysis of CUDAjectory v2
% 		performance improvement based on aforementioned analysis
% 		show with further performance analysis that our changes actually improved performance
% 	TODO Why are we doing it?
% 		mission analysis is a crucial aspect of mission design
% 			it influences all aspects of mission design
% 		especially for large-scale simulations with long propagation times
% 			these are used in 
% 	TODO What benefits do we think we'll gain after having done it?
% 		improved performance in trajectory calculations enable larger-scale simulations to be completed in less time
% 		enables Mission Analysis to iterate faster with higher quality data to ensure successful missions?
% 		TODO compose some sentences about how this enables better space missions
